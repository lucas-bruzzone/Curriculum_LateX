\documentclass[12pt]{article}
\usepackage[english]{babel}
\usepackage{cmbright}
\usepackage{enumitem}
\usepackage{fancyhdr}
\usepackage{fontawesome5}
\usepackage{geometry}
\usepackage{hyperref}
\usepackage[sf]{libertine}
\usepackage{microtype}
\usepackage{paracol}
\usepackage{supertabular}
\usepackage{titlesec}
\hypersetup{colorlinks, urlcolor=black, linkcolor=black}

% Geometry
\geometry{hmargin=1.75cm, vmargin=2.5cm}
\columnratio{0.65, 0.35}
\setlength\columnsep{0.05\textwidth}
\setlength\parindent{0pt}
\setlength{\smallskipamount}{8pt plus 3pt minus 3pt}
\setlength{\medskipamount}{16pt plus 6pt minus 6pt}
\setlength{\bigskipamount}{24pt plus 8pt minus 8pt}

% Style
\pagestyle{empty}
\titleformat{\section}{\scshape\LARGE\raggedright}{}{0em}{}[\titlerule]
\titlespacing{\section}{0pt}{\bigskipamount}{\smallskipamount}
\newcommand{\heading}[2]{\centering{\sffamily\Huge #1}\\\smallskip{\large{#2}}}
\newcommand{\entry}[4]{{{\textbf{#1}}} \hfill #3 \\ #2 \hfill #4}
\newcommand{\tableentry}[3]{\textsc{#1} & #2\expandafter\ifstrequal\expandafter{#3}{}{\\}{\\[6pt]}}

\begin{document}

\vspace*{\fill}

\begin{paracol}{2}

% Name & headline
\heading{Lucas Ricardo Duarte Bruzzone}{Engenheiro de Computação | Cientista de Dados}

\switchcolumn

% Identity card
\vspace{0.01\textheight}
\begin{supertabular}{ll}
  \footnotesize\faPhone & +55 35 98800 6634 \\
  \footnotesize\faEnvelope & \href{mailto:lucas.rbruzzone@gmail.com}{lucas.rbruzzone@gmail.com} \\
  \footnotesize\faLinkedin & \href{https://www.linkedin.com/in/lucas-bruzzone/}{linkedin.com/in/lucas-bruzzone/} \\
  \footnotesize\faGithub & \href{https://github.com/lucas-bruzzone}{lucas-bruzzone} \\
\end{supertabular}

\bigskip
\switchcolumn*

\section{Formação Acadêmica}

\entry{UFSCar}{Mestrado em Ciências da Computação}{São Carlos, Brasil}{2020 -- 2023}
\begin{itemize}[noitemsep,leftmargin=3.5mm,rightmargin=0mm,topsep=6pt]
  \item Aplicações em Aprendizado de Máquina.
  \item Desenvolvimento de um Classificador Fuzzy Incremental Aprimorado.
  \item Detecção de Novidades em Fluxos de Dados Multiclasse.
\end{itemize}

\medskip

\entry{IFSULDEMINAS}{Graduação em Engenharia de Computação}{Poços de Caldas, Brasil}{2015 -- 2019}

\switchcolumn

\section{Habilidades}
\begin{supertabular}{rl}
  \tableentry{\footnotesize\faCode}{Python \textperiodcentered{} Java \textperiodcentered{} Git}{}
  \tableentry{}{SQL \textperiodcentered{} Excel \textperiodcentered{} Ferramentas ETL}{}
  \tableentry{}{}{}

  \tableentry{\footnotesize\faLanguage}{Português \textperiodcentered{} Nativo}{}
  \tableentry{}{Espanhol \textperiodcentered{} Avançado}{}
  \tableentry{}{Inglês \textperiodcentered{} Intermediário}{}
\end{supertabular}

\switchcolumn*

\section{Experiência Profissional}


\entry{UFSCar}{Pesquisador Científico}{São Carlos, Brasil}{2022 -- Até o momento}
\begin{itemize}[noitemsep,leftmargin=3.5mm,rightmargin=0mm,topsep=6pt]
  \item Pesquisa em Detecção de Novidades em Fluxo de Dados.
  \item Desenvolvimento de Algoritmos.
  \item Aplicações Práticas em Ciências da Computação.
\end{itemize}

\medskip

\entry{TD2I}{Analista em BI e EPM}{Remoto, Latino América}{2019 -- 2021}
\begin{itemize}[noitemsep,leftmargin=3.5mm,rightmargin=0mm,topsep=6pt]
  \item Desenvolvimento de Projetos de Sustentação e Melhoria.
  \item Expertise em Ferramentas ETL como Talend, ODI e SSIS.
  \item Auxílio em Processos de Melhoria.
\end{itemize}

\switchcolumn


\section{Premiações}
\begin{supertabular}{rl}
  \tableentry{2023}{\textbf{Primeiro Lugar}}{}
  \tableentry{}{Data Challenge EY}{spaceafter}

\end{supertabular}

\end{paracol}

\vspace*{\fill}

\end{document}